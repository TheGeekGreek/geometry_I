\input{header.tex}

\begin{document}
\title{Geometry I - Summary}
\author{Yannis B\"{a}hni}
\address[Yannis B\"{a}hni]{University of Zurich, R\"{a}mistrasse 71, 8006 Zurich}
\email[Yannis B\"{a}hni]{\href{mailto:yannis.baehni@uzh.ch}{yannis.baehni@uzh.ch}}

\maketitle
\thispagestyle{fancy}

%\tableofcontents

\originalsectionstyle

\section{Topology}
\begin{definition}
	Let $X$ be a set. A \emph{topology on $X$} is a collection $\mathcal{T}$ of subsets of $X$ satisfying the following properties:
	
	\begin{enumerate}[label = (\roman*)]
		\item $X,\varnothing \in \mathcal{T}$.
		\item If $U_1,\dots,U_n \in \mathcal{T}$, then $U_1 \cap \dots \cap U_n \in \mathcal{T}$.
		\item If $(U_\alpha)_{\alpha \in A}$ is a family of elements of $\mathcal{T}$, then $\cup_{\alpha \in A} U_\alpha \in \mathcal{T}$.
	\end{enumerate}
\end{definition}

\vspace{1mm}

\begin{definition}
	Let $X$ be a set, and suppose $\mathcal{B}$ is a collection of subsets of $X$. Then $\mathcal{B}$ is a basis for some topology on $X$ if and only if it satisfies the following two conditions:
	\begin{enumerate}[label = (\roman*)]
		\item $\cup_{B \in \mathcal{B}} B = X$.
		\item If $B_1, B_2 \in \mathcal{B}$ and $x \in B_1 \cap B_2$, there exists an element $B_3 \in \mathcal{B}$ such that $x \in B_3 \subseteq B_1 \cap B_2$. 
	\end{enumerate}

	If so, there is a unique topology on $X$ for which $\mathcal{B}$ is a basis, called the \emph{topology generated by $\mathcal{B}$}.
\end{definition}

\vspace{1mm}

\begin{definition}
	If $d$ is a metric on the set $X$, then the collection of all $\varepsilon$-balls $B_\varepsilon(x)$, for $x \in X$ and $\varepsilon > 0$, is a basis for a topology on $X$, called the \emph{metric topology} induced by $d$.
\end{definition}

\vspace{1mm}

\begin{definition}
	If $X$ and $Y$ are topological spaces, a map $f: X \to Y$ is said to be \emph{continuous} if for every open subset $U \subseteq Y$, its preimage $f^{-1}(U)$ is open in $X$.
\end{definition}

\vspace{1mm}

\begin{proposition}
	Let $X$ be a Hausdorff space.

	\begin{enumerate}[label = (\alph*)]
		\item Every finite subset of $X$ is closed.
		\item If a sequence $(p_i)$ in $X$ converges to a limit $p \in X$, the limit is unique.
	\end{enumerate}
\end{proposition}

\vspace{1mm}

\begin{definition}
	A topological space is said to be \emph{second countable} if it admits a countable basis for its topology.
\end{definition}

\vspace{1mm}


\section{Geometry}
\begin{definition}
	Let $c: \intoo{a,b} \to \mathbb{R}^2$ be a regular curve. The \emph{planar curvature $\kappa$ of $c$} is defined to be

	\begin{equation}
		\kappa\del{t} := \frac{\det\del{c'\del{t},c''\del{t}}}{\norm[0]{c'\del{t}}^3}	
	\end{equation}
\end{definition}

\vspace{1mm}

\begin{proposition}
	Let $c: \intoo{a,b} \to \mathbb{R}^3$ be a regular curve. Then there is a reparametrization of $c$ that is a unit speed curve.
\end{proposition}

\begin{proof}
	Pick some point $t_0 \in \intoo{a,b}$. Define a function $q: \intoo{a,b} \to \mathbb{R}$ by $q(t) := \int_{t_0}^t \norm[0]{c'(s)} \d s$. The image of $q$ will be the interval $\intoo{d,e}$ where $d := \int_{t_0}^a \norm[0]{c'(s)}\d s$ and $e := \int_{t_0}^b \norm[0]{c'(s)}\d s$. Let $h: \intoo{d,e}\to\intoo{a,b}$ be the inverse function of $q$. The unit speed reparametrization $\tilde{c}: \intoo{d,e}\to \mathbb{R}^3$ is now given by $\tilde{c} := c \circ h$.
\end{proof}

\vspace{1mm}

\begin{definition}
	Let $c: \intoo{a,b} \to \mathbb{R}^3$ be a smooth curve. The \emph{length} of $c$ is defined to be

	\begin{equation}
		\Length(c) := \int_{a}^b \norm[0]{c'(s)} \d s
	\end{equation}
\end{definition}

\vspace{1mm}

\begin{definition}
	Let $c: \intoo{a,b} \to \mathbb{R}^3$ be a smooth curve. For each $t \in \intoo{a,b}$ such that $\norm[0]{c'(t)} \neq 0$ the \emph{unit tangent vector} to the curve at $t$ is the vector 

	\begin{equation}
		T(t) := \frac{c'(t)}{\norm[0]{c'(t)}}
	\end{equation}
\end{definition}

\vspace{1mm}

\begin{definition}
	Let $U \subseteq \mathbb{R}^2$ be open. A map $x \in C^\infty(U, \mathbb{R}^3)$ is a \emph{coordinate patch} if it is injective and if $x_1 \times x_2 \neq 0$ at all points of $U$. 
\end{definition}

\vspace{1mm}

\begin{lemma}
	Let $M \subseteq \mathbb{R}^3$ be a smooth surface and let $x: U \to M$ be a coordinate patch. For all $i,j = 1,2$, we have 

	\begin{equation}
		\begin{pmatrix}
			\Gamma^1_{ij}\\
			\Gamma^2_{ij}
		\end{pmatrix}
		= \frac{1}{2}\begin{pmatrix}
			g_{11} & g_{12}\\
			g_{21} & g_{22}
		\end{pmatrix}^{-1}
		\begin{pmatrix}
			\pd{g_{j1}}{u_i} + \pd{g_{i1}}{u_j} - \pd{g_{ij}}{u_1}\\
			\pd{g_{j2}}{u_i} + \pd{g_{i2}}{u_j} - \pd{g_{ij}}{u_2}
		\end{pmatrix}.
	\end{equation}
\end{lemma}
\end{document}
