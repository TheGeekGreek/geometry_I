%%%%%%%%%%%%%%%%%%%%%%%%%%%%%%%%%%%%%%%%%%%%%%%%%%%%%%%%%%%%%%%%%%%%%%%%%%
%Author:																 %
%-------																 %
%Yannis Baehni at University of Zurich									 %
%baehni.yannis@uzh.ch													 %
%																		 %
%Version log:															 %
%------------															 %
%06/02/16 . Basic structure												 %
%04/08/16 . Layout changes including section, contents, abstract.		 %
%%%%%%%%%%%%%%%%%%%%%%%%%%%%%%%%%%%%%%%%%%%%%%%%%%%%%%%%%%%%%%%%%%%%%%%%%%

%Page Setup
\documentclass[
	8pt, 
	oneside, 
	a4paper,
	reqno,
	final
]{amsart}

\usepackage[
	left = 3cm, 
	right = 3cm, 
	top = 3cm, 
	bottom = 3cm
]{geometry}

%Headers and footers
\usepackage{fancyhdr}
	\pagestyle{fancy}
	%Clear fields
	\fancyhf{}
	%Header right
	\fancyhead[R]{
		\footnotesize
		Yannis B\"{a}hni\\
		\href{mailto:yannis.baehni@uzh.ch}{yannis.baehni@uzh.ch}
	}
	%Header left
	\fancyhead[L]{
		\footnotesize
		MAT701: Geometry/Topology I\\
		HS16
	}
	%Page numbering in footer
	\fancyfoot[C]{\thepage}
	%Separation line header and footer
	\renewcommand{\headrulewidth}{0.4pt}
	%\renewcommand{\footrulewidth}{0.4pt}
	
	\setlength{\headheight}{19pt} 

%Title
\usepackage[foot]{amsaddr}
\usepackage{mathptmx}
\usepackage{xspace}
\makeatletter
\def\@textbottom{\vskip \z@ \@plus 1pt}
\let\@texttop\relax
\usepackage{etoolbox}
\patchcmd{\abstract}{\scshape\abstractname}{\textbf{\abstractname}}{}{}

%Switching commands for different section formats
%Mainsectionsytle
\newcommand{\mainsectionstyle}{%
  	\renewcommand{\@secnumfont}{\bfseries}
  	\renewcommand\section{\@startsection{section}{1}%
    	\z@{.5\linespacing\@plus.7\linespacing}{-.5em}%
    	{\normalfont\bfseries}}%
	\renewcommand\subsection{\@startsection{subsection}{2}%
    	\z@{.5\linespacing\@plus.7\linespacing}{-.5em}%
    	{\normalfont\bfseries}}%
	\renewcommand\subsubsection{\@startsection{subsubsection}{3}%
    	\z@{.5\linespacing\@plus.7\linespacing}{-.5em}%
    	{\normalfont\bfseries}}%
}
\newcommand{\originalsectionstyle}{%
\def\@secnumfont{\bfseries}%\mdseries
\def\section{\@startsection{section}{1}%
  \z@{.7\linespacing\@plus\linespacing}{.5\linespacing}%
  {\normalfont\bfseries\centering}}
}
%Formatting title of TOC
\renewcommand{\contentsnamefont}{\bfseries}
%Table of Contents
\setcounter{tocdepth}{3}

% Add bold to \section titles in ToC and remove . after numbers
\renewcommand{\tocsection}[3]{%
  \indentlabel{\@ifnotempty{#2}{\bfseries\ignorespaces#1 #2\quad}}\bfseries#3}
% Remove . after numbers in \subsection
\renewcommand{\tocsubsection}[3]{%
  \indentlabel{\@ifnotempty{#2}{\ignorespaces#1 #2\quad}}#3}
\let\tocsubsubsection\tocsubsection% Update for \subsubsection
%...

\newcommand\@dotsep{4.5}
\def\@tocline#1#2#3#4#5#6#7{\relax
  \ifnum #1>\c@tocdepth % then omit
  \else
    \par \addpenalty\@secpenalty\addvspace{#2}%
    \begingroup \hyphenpenalty\@M
    \@ifempty{#4}{%
      \@tempdima\csname r@tocindent\number#1\endcsname\relax
    }{%
      \@tempdima#4\relax
    }%
    \parindent\z@ \leftskip#3\relax \advance\leftskip\@tempdima\relax
    \rightskip\@pnumwidth plus1em \parfillskip-\@pnumwidth
    #5\leavevmode\hskip-\@tempdima{#6}\nobreak
    \leaders\hbox{$\m@th\mkern \@dotsep mu\hbox{.}\mkern \@dotsep mu$}\hfill
    \nobreak
    \hbox to\@pnumwidth{\@tocpagenum{\ifnum#1=1\bfseries\fi#7}}\par% <-- \bfseries for \section page
    \nobreak
    \endgroup
  \fi}
\AtBeginDocument{%
\expandafter\renewcommand\csname r@tocindent0\endcsname{0pt}
}
\def\l@subsection{\@tocline{2}{0pt}{2.5pc}{5pc}{}}
\def\l@subsubsection{\@tocline{2}{0pt}{4.5pc}{5pc}{}}
\makeatother

\advance\footskip0.4cm
\textheight=54pc    %a4paper
\textheight=50.5pc %letterpaper
\advance\textheight-0.4cm
\calclayout

%Font settings
%\usepackage{anyfontsize}
%Footnote settings
%\usepackage{mathptmx}
\usepackage{footmisc}
%	\renewcommand*{\thefootnote}{\fnsymbol{footnote}}
\usepackage{commath}
%Further math environments
%Further math fonts (loads amsfonts implicitely)
\usepackage{amssymb}
%Redefinition of \text
%\usepackage{amstext}
\usepackage{upref}
%Graphics
%\usepackage{graphicx}
%\usepackage{caption}
%\usepackage{subcaption}
%Frames
\usepackage{mdframed}
\allowdisplaybreaks
%\usepackage{interval}
\newcommand{\toup}{%
  \mathrel{\nonscript\mkern-1.2mu\mkern1.2mu{\uparrow}}%
}
\newcommand{\todown}{%
  \mathrel{\nonscript\mkern-1.2mu\mkern1.2mu{\downarrow}}%
}
\AtBeginDocument{\renewcommand*\d{\mathop{}\!\mathrm{d}}}
\renewcommand{\Re}{\operatorname{Re}}
\renewcommand{\Im}{\operatorname{Im}}
\DeclareMathOperator\Log{Log}
\DeclareMathOperator\Arg{Arg}
\DeclareMathOperator\sech{sech}
\DeclareMathOperator\Length{Length}
\DeclareMathOperator\Area{Area}
\DeclareMathOperator\tr{tr}
%\usepackage{hhline}
%\usepackage{booktabs} 
%\usepackage{array}
%\usepackage{xfrac} 
%\everymath{\displaystyle}
%Enumerate
\usepackage{tikz}
\usetikzlibrary{external}
\tikzexternalize % activate!
\usetikzlibrary{patterns}
\pgfdeclarepatternformonly{adjusted lines}{\pgfqpoint{-1pt}{-1pt}}{\pgfqpoint{40pt}{40pt}}{\pgfqpoint{39pt}{39pt}}%
{
  \pgfsetlinewidth{.8pt}
  \pgfpathmoveto{\pgfqpoint{0pt}{0pt}}
  \pgfpathlineto{\pgfqpoint{39.1pt}{39.1pt}}
  \pgfusepath{stroke}
}
\usepackage{enumitem}

\definecolor{anti-flashwhite}{rgb}{0.95, 0.95, 0.96}
%\renewcommand{\labelitemi}{$\bullet$}
%\renewcommand{\labelitemii}{$\ast$}
%\renewcommand{\labelitemiii}{$\cdot$}
%\renewcommand{\labelitemiv}{$\circ$}
%Colors
%\usepackage{color}
%\usepackage[cmtip, all]{xy}
%Theorems
\newtheoremstyle{bold}              	 %Name
  {}                              %Space above
  {}                              %Space below
  {\itshape}		                     %Body font
  {}                                     %Indent amount
  {\scshape}                             %Theorem head font
  {.}                                    %Punctuation after theorem head
  { }                                    %Space after theorem head, ' ', 
  										 %	or \newline
  {} 
\theoremstyle{bold}
\newmdtheoremenv[%
  backgroundcolor=anti-flashwhite,
  bottomline=false,
  topline=false,
  rightline=false,
  leftline=false]{definition}{Definition}[section]
 \newmdtheoremenv[%
  backgroundcolor=anti-flashwhite,
  bottomline=false,
  topline=false,
  rightline=false,
  leftline=false]{proposition}{Proposition}[section]
\newmdtheoremenv[%
  backgroundcolor=anti-flashwhite,
  bottomline=false,
  topline=false,
  rightline=false,
  leftline=false]{lemma}{Lemma}[section]
\newmdtheoremenv[%
  backgroundcolor=anti-flashwhite,
  bottomline=false,
  topline=false,
  rightline=false,
  leftline=false]{theorem}{Theorem}[section]
\newmdtheoremenv[%
  backgroundcolor=anti-flashwhite,
  bottomline=false,
  topline=false,
  rightline=false,
  leftline=false]{example}{Example}[section]\newtheorem*{definition*}{Definition}
%\newtheorem{definition}{Definition}[section]
\newtheorem*{lemma*}{Lemma}
%\newtheorem{lemma}{Lemma}[section]
\newtheorem{Proof}{Proof}[section]
%\newtheorem{proposition}{Proposition}[section]
\newtheorem{properties}{Properties}[section]
\newtheorem{corollary}{Corollary}[section]
\newtheorem*{theorem*}{Theorem}
%\newtheorem{theorem}{Theorem}[section]
%\newtheorem{example}{Example}[section]
\newtheorem*{remark*}{Remark}
\newtheorem{remark}{Remark}[section]
%German non-ASCII-Characters
%Graphics-Tool
%\usepackage{tikz}
%\usepackage{tikzscale}
%\usepackage{bbm}
%\usepackage{bera}
%Listing-Setup
%Bibliographie
\usepackage[backend=bibtex, style=alphabetic]{biblatex}
%\usepackage[babel, german = swiss]{csquotes}
\bibliography{Bibliography}
%PDF-Linking
%\usepackage[hyphens]{url}
\usepackage[bookmarksopen=true,bookmarksnumbered=true]{hyperref}
%\PassOptionsToPackage{hyphens}{url}\usepackage{hyperref}
\hypersetup{
  colorlinks   = true, %Colours links instead of ugly boxes
  urlcolor     = blue, %Colour for external hyperlinks
  linkcolor    = blue, %Colour of internal links
  citecolor    = blue %Colour of citations
}
%Weierstrass-P symbol for power set
\newcommand{\powerset}{\raisebox{.15\baselineskip}{\Large\ensuremath{\wp}}}


\begin{document}
\title{Geometry I - Summary}
\author{Yannis B\"{a}hni}
\address[Yannis B\"{a}hni]{University of Zurich, R\"{a}mistrasse 71, 8006 Zurich}
\email[Yannis B\"{a}hni]{\href{mailto:yannis.baehni@uzh.ch}{yannis.baehni@uzh.ch}}

\maketitle
\thispagestyle{fancy}

%\tableofcontents

\originalsectionstyle

\section{Topology}
\begin{definition}
	Let $X$ be a set. A \emph{topology on $X$} is a collection $\mathcal{T}$ of subsets of $X$ satisfying the following properties:
	
	\begin{enumerate}[label = (\roman*)]
		\item $X,\varnothing \in \mathcal{T}$.
		\item If $U_1,\dots,U_n \in \mathcal{T}$, then $U_1 \cap \dots \cap U_n \in \mathcal{T}$.
		\item If $(U_\alpha)_{\alpha \in A}$ is a family of elements of $\mathcal{T}$, then $\cup_{\alpha \in A} U_\alpha \in \mathcal{T}$.
	\end{enumerate}
\end{definition}

\vspace{1mm}

\begin{definition}
	Let $X$ be a set, and suppose $\mathcal{B}$ is a collection of subsets of $X$. Then $\mathcal{B}$ is a basis for some topology on $X$ if and only if it satisfies the following two conditions:
	\begin{enumerate}[label = (\roman*)]
		\item $\cup_{B \in \mathcal{B}} B = X$.
		\item If $B_1, B_2 \in \mathcal{B}$ and $x \in B_1 \cap B_2$, there exists an element $B_3 \in \mathcal{B}$ such that $x \in B_3 \subseteq B_1 \cap B_2$. 
	\end{enumerate}

	If so, there is a unique topology on $X$ for which $\mathcal{B}$ is a basis, called the \emph{topology generated by $\mathcal{B}$}.
\end{definition}

\vspace{1mm}

\begin{definition}
	If $d$ is a metric on the set $X$, then the collection of all $\varepsilon$-balls $B_\varepsilon(x)$, for $x \in X$ and $\varepsilon > 0$, is a basis for a topology on $X$, called the \emph{metric topology} induced by $d$.
\end{definition}

\vspace{1mm}

\begin{definition}
	If $X$ and $Y$ are topological spaces, a map $f: X \to Y$ is said to be \emph{continuous} if for every open subset $U \subseteq Y$, its preimage $f^{-1}(U)$ is open in $X$.
\end{definition}

\vspace{1mm}

\begin{proposition}
	Let $X$ be a Hausdorff space.

	\begin{enumerate}[label = (\alph*)]
		\item Every finite subset of $X$ is closed.
		\item If a sequence $(p_i)$ in $X$ converges to a limit $p \in X$, the limit is unique.
	\end{enumerate}
\end{proposition}

\vspace{1mm}

\begin{definition}
	A topological space is said to be \emph{second countable} if it admits a countable basis for its topology.
\end{definition}

\vspace{1mm}


\section{Geometry}
\begin{definition}
	Let $c: \intoo{a,b} \to \mathbb{R}^2$ be a regular curve. The \emph{planar curvature $\kappa$ of $c$} is defined to be

	\begin{equation}
		\kappa\del{t} := \frac{\det\del{c'\del{t},c''\del{t}}}{\norm[0]{c'\del{t}}^3}	
	\end{equation}
\end{definition}

\vspace{1mm}

\begin{proposition}
	Let $c: \intoo{a,b} \to \mathbb{R}^3$ be a regular curve. Then there is a reparametrization of $c$ that is a unit speed curve.
\end{proposition}

\begin{proof}
	Pick some point $t_0 \in \intoo{a,b}$. Define a function $q: \intoo{a,b} \to \mathbb{R}$ by $q(t) := \int_{t_0}^t \norm[0]{c'(s)} \d s$. The image of $q$ will be the interval $\intoo{d,e}$ where $d := \int_{t_0}^a \norm[0]{c'(s)}\d s$ and $e := \int_{t_0}^b \norm[0]{c'(s)}\d s$. Let $h: \intoo{d,e}\to\intoo{a,b}$ be the inverse function of $q$. The unit speed reparametrization $\tilde{c}: \intoo{d,e}\to \mathbb{R}^3$ is now given by $\tilde{c} := c \circ h$.
\end{proof}

\vspace{1mm}

\begin{definition}
	Let $c: \intoo{a,b} \to \mathbb{R}^3$ be a smooth curve. The \emph{length} of $c$ is defined to be

	\begin{equation}
		\Length(c) := \int_{a}^b \norm[0]{c'(s)} \d s
	\end{equation}
\end{definition}

\vspace{1mm}

\begin{definition}
	Let $c: \intoo{a,b} \to \mathbb{R}^3$ be a smooth curve. For each $t \in \intoo{a,b}$ such that $\norm[0]{c'(t)} \neq 0$ the \emph{unit tangent vector} to the curve at $t$ is the vector 

	\begin{equation}
		T(t) := \frac{c'(t)}{\norm[0]{c'(t)}}
	\end{equation}
\end{definition}

\vspace{1mm}

\begin{definition}
	Let $U \subseteq \mathbb{R}^2$ be open. A map $x \in C^\infty(U, \mathbb{R}^3)$ is a \emph{coordinate patch} if it is injective and if $x_1 \times x_2 \neq 0$ at all points of $U$. 
\end{definition}

\vspace{1mm}

\begin{lemma}
	Let $M \subseteq \mathbb{R}^3$ be a smooth surface and let $x: U \to M$ be a coordinate patch. For all $i,j = 1,2$, we have 

	\begin{equation}
		\begin{pmatrix}
			\Gamma^1_{ij}\\
			\Gamma^2_{ij}
		\end{pmatrix}
		= \frac{1}{2}\begin{pmatrix}
			g_{11} & g_{12}\\
			g_{21} & g_{22}
		\end{pmatrix}^{-1}
		\begin{pmatrix}
			\pd{g_{j1}}{u_i} + \pd{g_{i1}}{u_j} - \pd{g_{ij}}{u_1}\\
			\pd{g_{j2}}{u_i} + \pd{g_{i2}}{u_j} - \pd{g_{ij}}{u_2}
		\end{pmatrix}.
	\end{equation}
\end{lemma}
\end{document}
